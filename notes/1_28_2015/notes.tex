\documentclass{article}

\usepackage[margin=1in]{geometry}
\usepackage{graphicx}
\usepackage{amsmath}
\usepackage{amsfonts}


\author{Zachary Vogel}
\date{\today}
\title{Notes in ECEN 5623}

\begin{document}
\maketitle


\section*{Bureaucracy}
Liu and Layland Paper read by tuesday\\
Chapter 11 in the text.\\
Saturday at 5pm exercise 1 start.\\
Exercise 2 starts now.\\
Quiz 1 next Thursday.

\section*{Lecture}
Real-time service implementation.\\

Safe Resource Utilization Bound.\\

High-Level Design\\
Input and Output Hardware design and characterization.\\
Services: What are they?\\
System level methodology:\\
UML, and such.\\

Space Transportation System-shuttle example.\\
paper by Carlow for exercise 2.\\
Ascent and entry guidelines\\
Phases of flight divided into Major modes\\
each mode has real-time scheduling.\\
control and sensors in high frequency\\
navigation in medium frequency\\
guidance in low frequency\\
Ask Lucy Pao about this.\\

sirtf caltech. www.sirtf.caltech.edu\\
A CU example\\

RM assumptions:\\
All services requested on periodic basis, the period is constant\\
Completion time < Period\\
Service Requests are independent (no known phasing)\\
Run-time is known and deterministic.\\

RM constraints:\\
Deadline=Period by definition\\
Fixed Priority, preemptive, run-to-completion scheduling\\

critical instant: longest response time for a service occurs when al system serivces are requested simultaneously\\
No other shared resources- not in the paper, but it is a key assumption they make\\

derivation of RM LUB for 2 tasks\\
can you safely exceed the LUB\\

NOte that there can be up to the ciel$T2/T1$ releases of $S1$ in $T2$.\\


Next time: finish RM LUB derivation\\
discuss pitfalls\\
introduce extensions to overcome pitfalls.\\


Quiz will be largely over chapters 1 and 2. Look at methodology.

\end{document}
