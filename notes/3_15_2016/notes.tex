\documentclass{article}

\usepackage[margin=1in]{geometry}
\usepackage{graphicx}
\usepackage{amsmath}
\usepackage{amsfonts}


\author{Zachary Vogel}
\date{\today}
\title{Notes in ECEN 5623}

\begin{document}
\maketitle


\section*{Lecture}
read chapter 12 sections 6-11\\

form groups today for EX 5,6 and Final Project\\
3/21-25 Spring Break!\\
3/29 Exercise 4\\
4/5 Exercise 5\\

proposal is the core part of exercise 5.\\

\section*{Lecture}
ICT, IOT, OR, NOA, Di, CPI\\

Message Queues\\
Provide communication and synchronization\\
traditional message queue\\
meassage size, internal fragmentation\\
heap message queue\\
messages are pointers\\
message data in heap\\

nice thing is that once in the queue it can be seen and used atomically.\\
parallel processing speed up\\

Amdahl's law - speed up with \# of cores and parallel portion.\\
That's all for the I/O session.\\

Now we will talk more about implementation.\\
still grading exam 1\\
exercise 4 is last practice oriented 1\\
exercise 5 is to do the in-depth project description\\
exercise 6 will be to complete the projecct\\


\section*{BUS I/O}
VME and PCI busses\\
VME=(VESA MOdule Expansion) expansion on another bus\\
PCI= peripheral component interconnect\\
VME is asyncrhonous so you just have handshaking protocols\\
PCI has a synch clock\\
Bus arbitration built-in.\\
PCI also has bridging ability.\\



\end{document}
