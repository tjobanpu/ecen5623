\documentclass{article}

\usepackage[margin=1in]{geometry}
\usepackage{graphicx}
\usepackage{amsmath}
\usepackage{amsfonts}


\author{Zahary Vogel}
\date{\today}
\title{Notes in ECEN 5623}

\begin{document}
\maketitle


\section*{Day three bureacracy}
exercise 1 guid on the d2l page. Siewart CH 2.\\

\section*{Lecture}
review of rate monotonic scheduling example.\\
rate monotonic upper bound is sufficient not necessary.\\
least common multiple of periods.\\
feasible over LCM, feasible over all time\\

\subsection*{Complex Multi-service Systems}
multiple software services, synchronization and communication between services. Mutliple sensor actuator output interfaces, intermediate IO, shared memory, messaging.\\

multi-service pipelines diagram.\\

real time services, multi-service concurrency (multithreaded prolly)\\

to deal with problems they made real time service and cpu resource management theory.\\

why SW for hard real time systems?\\
cost, design time, software is easier to update, fpgas are costly, software is easier to fix\\
asic\\

real time is defined by there being deadline events.\\
without external events, why is it real time? book writer would argue it isn't\\

\subsection*{Operating System Overview}
intermediary interface between apps and hardware\\
facilitates application development\\

file system, scheduler, security, ectetera\\

os comes into existence.\\

structure types, monolithic (unix and linux), microkernel, most modern commercial adopt a hybrid approach.\\

embedded os, RTOS, EOS\\
another option is super-loop (sicklicle executives), this is okay for simple things\\
rtos, everything is a task scheduled based on demand.\\
jitter on response time is easier to control\\

RTOS aims at deterministic foremost, rather than throughput (consistency, reliability)\\
jitter is a measurement\\

two design approaches, event-driven and time-sharing\\
event driven(freertos): preemptive task scheduling, higher priority first, process data responsively
time-sharing(linux): lots of overhead, time-sharing, , tasks scheduled on regular clocked interrupt, round robin, more often switching. smoother multitasking\\

tasks can have 3 states, running, ready, blocked.\\
task scheduling, preemptive, cooperative, earliest deadline first, usually only one task per CPU at a time\\

RTOS's: lynx os, OSE, windows CE, freertos, arm keil\\

key terms: kernel task, Real time process, context switch, preemption, dispatch, Asymmetric multi-processing(AMP), Symmetric multi-processing(SMP), non-uniform memory access (NUMA), Simultaneous multi-threading (SMT), Flynn's taxonomy\\

Flynn's taxonomy - (single instruction, multi instruction) times (single data, multi-data)\\

taxonomy - method of classification\\

Flynn's computer architecture taxonomy\\

fair vs unfair scheduling: (completely fair scheduler) CFS in linux - fair\\

\end{document}
